%%%%%%%%%%%%%%%%%%%%%%%%%
%%%%%%%%%%%%%%%%%%%%%%%%%%%%%%%%%%%%%%%%%%%%%%%%%%%%%%%%%%%%%%%%
\begin{figure*}[h!]
     \centering
     \begin{tabular}{W{c}{0.18\textwidth}W{c}{0.18\textwidth}W{c}{0.18\textwidth}W{c}{0.18\textwidth}W{c}{0.18\textwidth}}
\begin{subfigure}[b]{0.35\linewidth}
 \begin{tikzpicture}[scale=0.80,line join=round, font=\small]
  \pie[radius= 1.6, rotate=180, text=inside] {10/Yes, 90/No}
 \end{tikzpicture}
\end{subfigure}
&
\begin{subfigure}[b]{0.35\linewidth}
 \begin{tikzpicture}[scale=0.80,line join=round, font=\small]
  \pie[radius= 1.6, rotate=180, text=inside] {20/Yes, 80/No}
 \end{tikzpicture}
\end{subfigure}
&
\begin{subfigure}[b]{0.35\linewidth}
 \begin{tikzpicture}[scale=0.80,line join=round, font=\small]
  \pie[radius= 1.6, rotate=180, text=inside] {10/Yes, 90/No}
 \end{tikzpicture}
\end{subfigure}
&
\begin{subfigure}[b]{0.35\linewidth}
 \begin{tikzpicture}[scale=0.80,line join=round, font=\small]
  \pie[radius= 1.6, rotate=180, text=inside] {30/Yes, 70/No}
 \end{tikzpicture}
\end{subfigure}
&
\begin{subfigure}[b]{0.35\linewidth}
 \begin{tikzpicture}[scale=0.80,line join=round, font=\small]
  \pie[radius= 1.6, rotate=180, text=inside] {20/Yes, 70/No, 10/Don't know}
 \end{tikzpicture}
\end{subfigure}
\\ 
\textbf{(A)} & \textbf{(B)} & \textbf{(C)} & \textbf{(D)} & \textbf{(E)} \\
\end{tabular}

\begin{tabular}{p{0.9\textwidth}}
\textbf{(A)} Was accessing and navigating the Blackboard platform and using the various functions easy for you?\\
\textbf{(B)} Do you create all the text portions in sizes large enough to be easily seen and distinguished?\\
\textbf{(C)} Do you know how to design course materials that has the accessibility guidelines for remote courses for deaf students?\\
\textbf{(D)} Are you able to access all the contents of your classes on Blackboard?\\
\textbf{(E)} If tables and pictures are added, do you add an explanation of the picture and table?
\end{tabular}
%  \floatfoot{\textbf{(a)} Was accessing and navigating the Blackboard platform and using the various functions easy for you?}\\
%  \floatfoot{\textbf{(b)} Do you create all the text portions in sizes large enough to be easily seen and distinguished?}\\
%  \floatfoot{\textbf{(c)} Do you know how to design course materials that has the accessibility guidelines for remote courses for deaf students?}\\
%  \floatfoot{\textbf{(d)} Are you able to access all the contents of your classes on Blackboard?}\\
%  \floatfoot{\textbf{(e)} If tables and pictures are added, do you add an explanation of the picture and table?}\\
%\vspace{-0.4cm}
 \caption{Responses to questions regarding to accessibility aspect.}
%\vspace{-0.4cm}
\label{fig:A11y_Questions}
\end{figure*}

%%%%%%%%%%%%%%%%%%%%%%%%%%%%%%%%%%%%%%%%%%%%%%%%%%%%%%%%%%%%%%%%

% \begin{figure}[h!]
% \begin{tikzpicture}
%     \node [anchor=north west] (imgA) at (.05\linewidth, .050\linewidth){\includegraphics[width=.5\linewidth]{figA.pdf}};
%     \draw [anchor=north west] (0., .000\linewidth) node {(A)};
%     \node [anchor=north west] (imgB) at (.05\linewidth, .500\linewidth){\includegraphics[width=.\linewidth]{figB.pdf}};
%     \draw [anchor=north west] (0., .500\linewidth) node {(A)};
% \end{tikzpicture}\
% \end{figure}

\begin{figure}
\subfloat[Flower one.]{\includegraphics[width=.45\linewidth]{example-image-a}\label{fig:f2}} 
\subfloat[Flower one.]{ \includegraphics[width=.45\linewidth]{example-image-b}\label{fig:f2}}
\\
\subfloat[Flower one.]{\includegraphics[width=.45\linewidth]{example-image-a}\label{fig:f2}} 
\subfloat[Flower one.]{ \includegraphics[width=.45\linewidth]{example-image-b}\label{fig:f2}}

\caption{MWE to demonstrate how to place to images side-by-side}
\end{figure}